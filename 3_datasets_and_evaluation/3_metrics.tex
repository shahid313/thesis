\section{Evaluation Metrics}

6.5.2 Evaluation Measures
Evaluation is done by comparing the system output with the reference annotations
available for the test data. Metrics used in detection and classification of sound
scenes and sound events include accuracy, precision, recall, F-score, receiver
operating characteristic (ROC) curve, area under the curve (AUC), acoustic event
error rate (AEER), or simply error rate (ER). There is no consensus over which
metric is universally good for measuring performance of sound event detection, as
they each reflect different perspectives on the ability of the system.
6.5.2.1 Intermediate Statistics
Many performance measures rely on trials, the notion of atomic opportunities for
the system to be correct or make a mistake, and the metrics are calculated based on
counts of the correct predictions and different types of errors made by the system.
These counts are referred to as intermediate statistics and are defined depending on
the evaluation procedure. Given class c they are defined as follows:
– True positive: A correct prediction, meaning that the system output and the
reference both indicate class c present or active.
– True negative: The system output and the reference both indicate class c not
present or inactive.
– False positive or insertion: The system output indicates class c present or active,
while the reference indicates class c not present or inactive.
– False negative or deletion: The system output indicates class c is not present or
inactive, while the reference indicates class c present or active.
A false positive can appear at the same time as a false negative, when the system
output indicates class c while the reference indicates class g. In this situation, some
metrics consider that the system makes a single error—a substitution—instead of
two separate errors. More details on how substitutions are defined for specific cases
are presented with the description of the metrics that use them.
The intermediate statistics are calculated based on the trial-wise comparison
between the system output and the available reference for the test data [47].
Individual trials can be formed by specific items identified in the annotations, or
simply from fixed length intervals. The fixed length intervals can be any small
temporal unit, such as a frame of 20–100 ms, similar to the typical audio analysis
frame [39, 53], or a longer interval such as 1 s [24]. In item-level comparisons
the item can be an entire audio clip from start to end, or a sound event instance.
Corresponding metrics are referred to as segment-based metrics and item-based or
event-based [32] metrics in the case of sound events.
Acoustic scene classification is usually a single-label multiclass problem, and
the resulting intermediate metrics reflect whether the single true class is correctly
recognized for each example. In this task there is no role for substitutions, and each
erroneous output is counted; there is no distinction between false positives and false
negatives. In scene segmentation the intermediate statistics can be counted in short
time intervals or based on segmentation boundaries.
In sound event detection, the choice of measurement determines the interpretation of the result: With a segment-based metric, the performance shows how
well the system correctly detects the temporal regions where a sound event is
active; with an event-based metric, the performance shows how well the system
is able to detect event instances with correct onset and offset. If we consider the
scenario of polyphonic sound event detection, the segment-based metric essentially
splits the duration of the test audio into fixed length segments that have multiple
associated labels, reflecting the sound events active anywhere in the given segment.
In this respect, evaluation verifies if the system output and reference coincide in the
assigned labels, and the length of the segment determines the temporal resolution of
the evaluation. Event-based metrics compare event instances one to one. Since the
time extents of the events detected by the system may not exactly match the ground
truth, a common approach is to allow a time misalignment threshold, either as a fixed
value (e.g., 100 ms) or as a fixed proportion of the total event duration (e.g., 50%)
[53]. Time thresholds can be applied to onset times only, or to both onset and offset
times. True negatives (i.e., correctly recording that no event occurred at a given
time) do not occur in this kind of evaluation, as no instance-based true negatives
can be counted. Figure 6.6 illustrates the comparison between a system output
and the reference in segment-based and event-based evaluations, for obtaining the
intermediate statistics.
6.5.2.2 Metrics
Measures of performance are calculated based on accumulated values of the intermediate statistics. We denote by TP, TN, FP, and FN the sums of the true positives,
true negatives, false positives, and false negatives accumulated throughout the test
data. With same convention, we will use S for the total number of substitutions, I
for insertions, and D for deletions.
Considering a simple binary classification, where we compare the reference and
the system output for class c, we can construct the contingency table, or confusion
matrix presented in Fig. 6.7. Based on the total counts of the intermediate statistics,
many different measures can be derived, of which the most commonly used are
recall (R, also known as true positive rate (TPR) or sensitivity), precision (P), false
positive rate (FPR), specificity, and accuracy (ACC). These measures are presented
and defined in Fig. 6.7.
When dealing with a multiclass problem, accumulation of intermediate statistics
can be performed either globally or separately for each class [49], resulting in
overall metrics calculated accordingly as instance-based or class-based. Highly
unbalanced classes or individual class performance can result in very different
overall performance when calculated with the two methods. In instance-based
averaging, also called micro-averaging, intermediate statistics are accumulated over
the entire data. Overall performance is calculated based on these, resulting in metrics
with values that are most strongly affected by the performance on the most common
classes in the considered problem. In class-based averaging (the results of which
are also known as balanced metrics), also called macro-averaging, intermediate
statistics are accumulated separately for each category (scene or event class), and
used to calculate class-wise metrics. Overall performance is then calculated as the
average of class-wise performance, resulting in values that emphasize the system
behavior on the smaller classes in the considered problem. A hybrid app
calculate class-based metrics, then to use nonuniform weighting to combine them to
create an average that reflects some relative importance of the different classes that
may be different from their frequency of occurrence in the test data.
Accuracy Accuracy measures how often the classifier makes the correct decision,
as the ratio of correct system outputs to total number of outputs. As shown in
Fig. 6.7, accuracy is calculated as:
ACC D
TP C TN
TP C TN C FP C FN
(6.1)
Accuracy has the advantage of offering a simple measure of the ability of the system
to take the correct decision; it is the most-used metric in sound scene and sound
event classification, being straightforward to calculate and easy to understand. It
has, however, a critical disadvantage of being influenced by the class balance: for
rare classes (i.e., where TP C FN is small), a system can have a high proportion
of true negatives even if it makes no correct predictions, leading to a paradoxically
high accuracy value. Accuracy does not provide any information about the error
types (i.e., the balance of FP and FN); yet, in many cases these different types of
error have very different implications.
Precision, Recall, and F-Score Precision, recall, and F-score were introduced in
[42] in the context of information retrieval, but have found their way into measuring
performance in other applications. Precision and recall are also defined in Fig. 6
1
P D
TP
TP C FP
; R D
TP
TP C FN
(6.2)
and based on them, balanced F-score is calculated as their harmonic mean:
F D
2
1=P C 1=R D
2PR
P C R
(6.3)
Precision and recall are the preferred metrics in information retrieval, but are used
also in classification under the names positive prediction value and sensitivity. In
detection theory terms, recall is equal to true positive rate, but precision has no
simple equivalent.
F-score has the advantage of being a familiar and well understood metric. Its
main drawback is that its value is strongly influenced by the choice of averaging
and the data balance between classes: in instance-based averaging the performance
on common classes dominates, while in class-based averaging (balanced metrics) it
is necessary to at least ensure presence of all classes in all folds in the test data, to
avoid cases when recall is undefined (when TP C FN D 0); estimates of metrics on
classes with very few examples are also intrinsically noisy. Any dataset of real-world
recordings will most likely have unbalanced event classes; therefore, the experiment
setup must be built with the choice of metric in mind.
Average Precision Because precision and recall rely on hard decisions made for
each trial, they typically depend on a threshold applied to some underlying decision
variable, such as a distance from a decision boundary or the output of a neural
network. Lowering the threshold will increase likelihood of accepting both positive
and negative examples, improving recall but in many cases hurting precision. Fmeasure combines these values at a single threshold in an attempt to balance this
tradeoff, but a fuller picture is provided by plotting precision as a function of
recall over the full range of possible thresholds—the precision–recall (P-R) curve.
Figure 6.8 shows an example of P-R curve for a binary classification problem.

In this work we used the Error Rate (ER) as primary evaluation metric to ensure comparability with the reference systems. In particular, for the evaluations on the TUT-SED 2016 and 2017 datasets we consider a segment-based ER with a one-second segment length, while for the TUT-Rare 2017 the evaluation metric is event-based error rate calculated using onset-only condition with a collar of 500 ms. 
In the segment-based ER the ground truth and system output are compared in a fixed time grid, thus sound events are marked as active or inactive in each segment. For the event-based ER the ground truth and system output are compared at event instance level.

ER score is calculated in a single time frame of one second length from intermediate statistics, i.e., the number of substitutions ($S(t_1)$), insertions ($I(t_1)$), deletions ($D(t_1)$) and active sound events from annotations ($N(t_1)$) for a segment $t_1$. Specifically:
\begin{enumerate}
	
	\item Substitutions $S(t_1)$ are the number of ground truth events for which we have a false positive and one false negative in the same segment; %, thus: $S(t_1) = \min(FN(t_1),FP(t_1))$;
	
	\item Insertions $I(t_1)$ are events in system output that are not present in the ground truth, thus the false positives which cannot be counted as substitutions;%: $I(t_1) = \max(0,FN(t_1)-FP(t_1))$;
	
	\item Deletions $D(t_1)$ are events in ground truth that are not correctly detected by the system, thus the false negatives which cannot be counted as substitutions;%: $D(t_1)= \max(0,FP(t_1)-FN(t_1))$.
	
\end{enumerate}
These intermediate statistics are accumulated over the segments of the whole test set to compute the evaluation metric ER. Thus, the total error rate is calculated as:
\begin{equation}
ER = \frac{\sum_{t_1=1}^{T} S(t_1) + \sum_{t_1=1}^{T} I(t_1) + \sum_{t_1=1}^{T} D(t_1)}{\sum_{t_1=1}^{T} N(t_1)},
\end{equation}
where $T$ is the total number of segments $t_1$.

If there are multiple scenes in the dataset, such as in the TUT-SED 2016, evaluation metrics are calculated for each scene separately and then the results are presented as the average across the scenes.
A detailed and visualized explanation of segment-based ER score in multi label setting can be found in \cite{mesaros2016metrics}.